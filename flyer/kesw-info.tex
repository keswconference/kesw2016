\documentclass[a4paper, 10pt]{article}
\usepackage[T2A,T1]{fontenc}
\usepackage[utf8]{inputenc}
\usepackage[russian,english]{babel}
\usepackage{graphicx}
\usepackage{amsmath}
\usepackage{color}
\usepackage{epstopdf}
\usepackage{url}

\renewcommand{\familydefault}{\sfdefault}
\newcommand{\rus}[1]{\foreignlanguage{russian}{#1}}
\renewcommand{\skip}{\vspace{1ex}}
\renewcommand{\bullet}{\ensuremath{\star} }

\pagenumbering{gobble}

\begin{document}

\begin{minipage}{.80\textwidth}
\begin{center}

\Large{\bf 6\textsuperscript{th} International Knowledge Engineering and Semantic Web Conference 
\\ (KESW 2015)}

\skip\skip

\large{\bf \rus{Москва, Российская Венчурная Компания}, 30.09.2015--02.10.2015 
\\
\url{http://2015.kesw.ru}}
\end{center}

\end{minipage}
\hfill
\begin{minipage}{.21\textwidth}

\includegraphics[width=\textwidth]{kesw}

\end{minipage}

\skip\skip

\begin{otherlanguage*}{russian}
\noindent
\textbf{KESW} --- это ведущая научная конференция по тематике Semantic Web,
проводящаяся на территории СНГ и Восточной Европы. Мы приглашаем к участию всех интересующихся научными и практическими аспектами семантических технологий, в частности, следующими темами:
\end{otherlanguage*}

\skip
\skip

\begin{minipage}[t]{.35\textwidth}
\begin{otherlanguage*}{russian}
\noindent
\bullet \textbf{Онтологии}

\bullet \textbf{Представление знаний}

\bullet \textbf{Визуализация знаний}

\end{otherlanguage*}
\end{minipage}
\hfill
\begin{minipage}[t]{.30\textwidth}

\bullet \textbf{Linked Data}

\bullet \textbf{Big Data}

\bullet\textbf{OWL/RDF/SPARQL}

\end{minipage}
\hfill
\begin{minipage}[t]{.28\textwidth}

\begin{otherlanguage*}{russian}
\bullet \textbf{Лингвистика}

\bullet \textbf{Schema.org}

\bullet \textbf{Открытые данные}

\end{otherlanguage*}

\end{minipage}

\skip
\skip

\begin{otherlanguage*}{russian}
\noindent \textbf{KESW} --- это:
\end{otherlanguage*}

\skip\skip

\begin{otherlanguage*}{russian}

\noindent\bullet международные эксперты в качестве приглашенных докладчиков;

\noindent\bullet публикация трудов в Springer~CCIS с индексацией в Scopus;

\noindent\bullet конструктивное рецензирование статей по западным образцам;

\noindent\bullet умеренный оргвзнос и финансовая поддержка студентов и аспирантов.

\end{otherlanguage*}

\skip\skip


\begin{otherlanguage*}{russian}
\noindent На данный момент подтверждены \textbf{пленарные доклады} от:

\noindent\bullet Markus Stocker, университет Восточной Финляндии

\noindent\bullet Юлия Тихоход, \textcolor{red}{\textbf{Я}}ндекс
\end{otherlanguage*}

\skip\skip

\begin{otherlanguage*}{russian}
\noindent 
На KESW 2015 планируется отдельная секция, посвященная \textbf{открытым данным,
науке и образованию}, под руководством  Ирины Радченко (ИТМО). 
%Принимаются статьи,
%посвящённые работе с открытыми источниками образовательных и научных данных,
%семантическими стандартами в образовании и науке, журналистике данных и смежным
%темам.
\end{otherlanguage*}

\skip\skip

\begin{minipage}{.10\textwidth}
\includegraphics[width=\textwidth]{information}
\end{minipage}
\hfill
\begin{minipage}{.82\textwidth}
\begin{otherlanguage*}{russian}
\noindent Язык KESW -- \textbf{английский}. 
К рецензированию допускаются только \textbf{полные} версии статей на английском языке.
Все тематические секции, в том числе пленарные доклады, устные и стендовые доклады, также будут проводиться на английском языке.
\end{otherlanguage*}
\end{minipage}

\skip\skip

\begin{otherlanguage*}{russian}
\noindent\textbf{Сроки приёма работ}

\skip

\noindent
\bullet Регистрация аннотаций: 4 мая 2015 г.

\noindent
\bullet Приём статей на рецензирование: 10 мая 2015 г.

\noindent
\bullet Решение о принятии статей к публикации: 21 июня 2015 г.

\noindent
\bullet Приём окончательных версий статей: 4 июля 2015 г.

\skip

\hspace*{-\parindent}%
\begin{minipage}{.80\textwidth}
\begin{otherlanguage*}{russian}

\textbf{Организаторы}

\skip

\noindent
\bullet Дмитрий Муромцев (ИТМО, РФ), оргкомитет.

\bullet Павел Клинов (Complexible, США), программный комитет.

\bullet Ирина Радченко (ИТМО, РФ), секция ``Открытые Данные'' 

\bullet Максим Колчин (ИТМО, РФ), Web-дизайн и PR.

\end{otherlanguage*}
\end{minipage}
\begin{minipage}{.20\textwidth}
\includegraphics[width=0.8\textwidth]{qrcode}

\footnotesize{\url{2015.kesw.ru}}
\end{minipage}
\hfill


\vfill

\noindent Более подробная информация содержится на сайте конференции. Вопросы можно задавать по адресу \url{info@kesw.ru} 
или \url{pavel@complexible.com}.

%\skip

%\noindent
%\includegraphics[width=0.3\textwidth]{logo_itmo.png}\hfill 
%\includegraphics[width=0.4\textwidth]{springer.png}

\end{otherlanguage*}

\end{document}
